% various imports 
\documentclass{article}
\usepackage{graphicx}
\usepackage{amsmath}
\usepackage{amsfonts}
\usepackage{amssymb}
\usepackage{mathrsfs}
\usepackage{prftree}
\usepackage{float}
\usepackage{syntax}
\usepackage{cmll}

\usepackage[margin=1in]{geometry}

% define multiset notation
\newcommand*{\ldblbrace}{\{\mskip-5mu\{}
\newcommand*{\rdblbrace}{\}\mskip-5mu\}}

\setlength\parindent{0pt}

\begin{document}

\title{LinML}
\author{Ada Vienot}
\date{}

\maketitle

\section{Typing}

The types in LinML have the following syntax :

\begin{figure}[!h]
    \centering
    \begin{minipage}{0.7\linewidth}
        \setlength{\grammarindent}{3.8em}
        \begin{grammar}
            \let\syntleft\relax
            \let\syntright\relax
            <$\tau, \sigma$> ::= 
                % constant
                $0 \mid 1 \mid \top$ \hfill (type constants)
                % type variables
                \alt $a$ \hfill (type variables)
                % arrows 
                \alt $\tau \multimap \sigma$ \hfill (linear abstraction)
                \alt $\tau \to \sigma$ \hfill (unchecked abstraction)
                % binary constructs 
                \alt $\tau + \sigma$ \hfill (additive disjunction)
                \alt $\tau * \sigma$ \hfill (multiplicative conjunction)
                \alt $\tau \with \sigma$ \hfill (multiplicative disjunction)
                % bang 
                \alt $\oc\tau$ \hfill (of course)
        \end{grammar}
    \end{minipage}
    \caption{LinML types}
    \label{types}
\end{figure}

\subsection{Typing rules}

We define the inductive predicate $\operatorname{exp}$ over types as follows :

$$
\operatorname{exp}(\tau) = 
\left \{
\begin{aligned}
    \mathrm{true} \qquad &\text{if } \tau = * \text{ or } \tau = !\sigma\\
    \mathrm{false} \qquad &\text{elsewise}
\end{aligned}
\right.
$$

In all the following, typing contexts $\Gamma \Vert \Delta$ are to be seen as a pair with :
\begin{itemize}
    \item $\Gamma$, called a \textit{linear context}, is a multiset of pairs $(x : \tau)$, with $x$ a variable and $\tau$ a type such that $\operatorname{exp}(\tau) = \mathrm{false}$
    
    \item $\Delta$, called an \textit{exponential context}, is a set of pairs $(x : \tau)$, with $x$ a variable and $\tau$ a type such that $\operatorname{exp}(\tau) = \mathrm{true}$    
\end{itemize}

Since $\operatorname{exp}(\tau)$ implies the existence of context weakening and contraction rules for $\tau$, we treat the context $\Delta$ as the kind of contexts we manipulate in intuitionistic sequent calculus, that is : they can be duplicated, erased, and one binding is allowed to erase another. This is not the case of the context $\Gamma$ containing linear bindings.

\subsubsection{Terms}

Let $t, u$ be terms, $\beta, \beta'$ booleans, $\tau, \sigma$ types, and $(\Gamma \parallel \Delta), (\Gamma' \parallel \Delta')$ typing contexts. The typing judgements for terms have the following shape :
$$
\Gamma \parallel \Delta \vdash t : \tau \Rightarrow \Gamma' \rtimes \beta
$$

$\Gamma'$ is the linear context where all the necessary bindings to type $t : \tau$ have been consumed. 

$\beta \in \{\top, \bot\}$ is the called the "modality of the judgement". \vspace{\baselineskip}

A program $p$ is well typed of type $\tau$ iff : 
$$
\varnothing \parallel \varnothing \vdash p : \tau \Rightarrow \varnothing \rtimes \beta
$$

The typing rules are the following:

\begin{figure}[H]
\centering

    % constants
    \begin{tabular}{ll}
        % rule for one 
        \prftree[rule]{\scriptsize ($1$)} { \Gamma \parallel \Delta \vdash * : 1 \Rightarrow \Gamma \rtimes \bot } &
        % rule for top
        \prftree[rule]{\scriptsize ($\top$)} { \Gamma \parallel \Delta \vdash \langle \rangle : \top \Rightarrow \Gamma \rtimes \top }
    \end{tabular} \\[\baselineskip]

    % zero 
    \begin{tabular}{l}
        % bang
        \prftree[rule]{\scriptsize ($0$)} 
            { \Gamma \parallel \Delta \vdash t : 0 \Rightarrow \Gamma' \rtimes \beta' }
            { \Gamma \parallel \Delta \vdash \textbf{refute } \tau \textbf{ with } t :  \tau \Rightarrow \Gamma' \rtimes \top }
    \end{tabular}\\[\baselineskip]

    % variable 
    \begin{tabular}{ll}
        % variable
        \prftree[rule]{\scriptsize (vlin)} 
            {  (x : \tau) \in \Gamma }
            { \quad \neg\operatorname{exp}(\tau) }
            { \Gamma \parallel \Delta \vdash x : \tau \Rightarrow \Gamma \setminus (x : \tau) \rtimes \bot }&
        % variable-exp
        \prftree[rule]{\scriptsize (vexp)} 
            {  (x : \tau) \in \Delta }
            { \quad \operatorname{exp}(\tau) }
            { \Gamma \parallel \Delta \vdash x : \tau \Rightarrow \Gamma \rtimes \bot }
    \end{tabular} \\[1.5\baselineskip]

    % times
    \begin{tabular}{l}
        \prftree[rule]{\scriptsize ($\otimes$)} 
            { \Gamma \parallel \Delta \vdash t : \tau \Rightarrow \Gamma' \rtimes \beta' }
            { \quad \Gamma' \parallel \Delta \vdash u : \sigma \Rightarrow \Gamma'' \rtimes \beta'' }
            { \Gamma \parallel \Delta \vdash (t, u) : \tau * \sigma \Rightarrow \Gamma'' \rtimes \beta' \vee \beta'' }
    \end{tabular} \\[1.5\baselineskip]

    % with
    \begin{tabular}{l}
        \prftree[rule]{\scriptsize ($\with$)} 
            { \Gamma \parallel \Delta \vdash t : \tau \Rightarrow \Gamma' \rtimes \beta' }
            { \quad \Gamma \parallel \Delta \vdash u : \sigma \Rightarrow \Gamma'' \rtimes \beta'' }
            { \quad \beta' \implies \Gamma' \subseteq \Gamma'' }
            { \quad \beta'' \implies \Gamma'' \subseteq \Gamma'  }
            { \Gamma \parallel \Delta \vdash \langle t, u \rangle : \tau \with \sigma \Rightarrow \Gamma' \cap \Gamma'' \rtimes \beta' \wedge \beta'' }
            % { \Gamma \vdash t : \tau \Rightarrow \Gamma' }
            % { \quad \Gamma \vdash u : \sigma \Rightarrow \Gamma'' }
            % { (\Gamma \setminus \Gamma') \cap (\Gamma \setminus \Gamma'') = \varnothing }; \Delta
            % { \Gamma \vdash \langle t, u \rangle : \tau \with \sigma \Rightarrow \Gamma' \cap \Gamma'' }
    \end{tabular} \\[1.5\baselineskip]

    % disjunction
    \begin{tabular}{ll}
        % l 
        \prftree[rule]{\scriptsize ($\oplus$-l)} 
            { \Gamma \parallel \Delta \vdash t : \sigma \Rightarrow \Gamma' \rtimes \beta' }
            { \Gamma \parallel \Delta \vdash (t :> \_ + \tau) : \sigma + \tau \Rightarrow \Gamma' \rtimes \beta' }&
        % r
        \prftree[rule]{\scriptsize ($\oplus$-r)} 
            { \Gamma \parallel \Delta \vdash u : \tau \Rightarrow \Gamma' \rtimes \beta' }
            { \Gamma \parallel \Delta \vdash (u :> \sigma + \_) : \sigma + \tau \Rightarrow \Gamma' \rtimes \beta' }
    \end{tabular} \\[1.5\baselineskip]

    % bang 
    \begin{tabular}{lll}
        % bang
        \prftree[rule]{\scriptsize ($\oc$)} 
            { \varnothing \parallel \Delta \vdash t : \tau \Rightarrow \varnothing \rtimes \beta }
            { \quad \Gamma = \Gamma' }
            { \Gamma \parallel \Delta \vdash \oc t : \oc \tau \Rightarrow \Gamma \rtimes \bot }&
        % bang
        \prftree[rule]{\scriptsize ($\oc$-dig)} 
            { \Gamma \parallel \Delta \vdash \oc t : \oc \tau \Rightarrow \Gamma' \rtimes \beta' }
            { \Gamma \parallel \Delta \vdash \oc \oc t : \oc \tau \Rightarrow \Gamma' \rtimes \beta' }
    \end{tabular}\\[1.5\baselineskip]
    
    % lolli
    \begin{tabular}{l}
        % lollipop
        \prftree[rule]{\scriptsize ($\multimap$)} 
            { (\Gamma \setminus x) \cdot (x:\tau) \parallel \Delta \setminus x \vdash t : \sigma \Rightarrow \Gamma' \rtimes \beta' }
            { \quad x \in \Gamma' \implies \beta' }
            { \quad \neg\operatorname{exp}(\tau)}
            { \quad \tau \neq \top }
            % { \tau \neq \top }
            { \Gamma \parallel \Delta \vdash \textbf{fun } (x:\tau) \multimap t : \tau \multimap \sigma \Rightarrow \Gamma' \setminus (x : \tau) \rtimes \beta' }
    \end{tabular}\\[1\baselineskip]

    \begin{tabular}{l}
        % lollipop
        \prftree[rule]{\scriptsize ($\multimap$-!)} 
            { \Gamma \setminus x \parallel (\Delta \setminus x) \cup (x:\tau) \vdash t : \sigma \Rightarrow \Gamma' \rtimes \beta' }
            { \quad \operatorname{exp}(\tau)}
            % { \tau \neq \top }
            { \Gamma \parallel \Delta \vdash \textbf{fun } (x:\tau) \multimap t : \tau \multimap \sigma \Rightarrow \Gamma' \rtimes \beta' }
    \end{tabular}\\[1\baselineskip]

    \begin{tabular}{l}
        % lollipop app
        \prftree[rule]{\scriptsize ($\multimap$-app)} 
            { \Gamma \parallel \Delta \vdash t : \sigma \multimap \tau \Rightarrow \Gamma' \rtimes \beta' }
            { \quad \Gamma' \parallel \Delta \vdash t' : \sigma \Rightarrow \Gamma'' \rtimes \beta'' }
            { \Gamma \parallel \Delta \vdash t t' : \tau \Rightarrow \Gamma'' \rtimes \beta' \vee \beta' }
    \end{tabular}\\[1\baselineskip]

    
    \begin{tabular}{l}
        % lollipop app
        \prftree[rule]{\scriptsize ($\multimap$-!app)} 
            { \Gamma \parallel \Delta \vdash t : \ !(\sigma \multimap \tau) \Rightarrow \Gamma \rtimes \bot }
            { \quad \Gamma \parallel \Delta \vdash t' : \sigma \Rightarrow \Gamma' \rtimes \beta' }
            { \Gamma \parallel \Delta \vdash t t' : \tau \Rightarrow \Gamma'' \rtimes \beta' }
    \end{tabular}\\[1.5\baselineskip]

    % arrow
    % \begin{tabular}{ll}
    %     % arrow
    %     \prftree[rule]{\scriptsize ($\to$)} 
    %         { \Gamma, x:\oc\tau \vdash t : \sigma }
    %         { \Gamma \vdash \textbf{fun } (x:\tau) \to t : \tau \to \sigma }&
    %     % arrow app 
    %     \prftree[rule]{\scriptsize ($\to$-app)} 
    %     { \Gamma \vdash t : \sigma \to \tau}
    %     { \quad \Gamma' \vdash t' : \oc\sigma }
    %     { \Gamma, \Gamma' \vdash t t' : \tau }
    % \end{tabular}\\[1.5\baselineskip]

    % give
    \begin{tabular}{l}
        \prftree[rule]{\scriptsize (give)} 
            { \Gamma \parallel \Delta \vdash t : \sigma \Rightarrow \Gamma' \rtimes \beta' }
            { \enskip \Gamma' \cdot (x : \sigma) \parallel \Delta \setminus x \vdash t' : \tau \Rightarrow \Gamma'' \rtimes \beta'' }
            { \enskip x \in \Gamma'' \implies \beta'' }
            {  x \notin \Gamma' }
            { \neg \operatorname{exp}(\sigma) } 
            {  \sigma \neq \top }
            { \Gamma \parallel \Delta \vdash \textbf{give } x = t \textbf{ in } t' : \tau \Rightarrow (\Gamma'' \setminus x) \rtimes \beta' \vee \beta'' }
    \end{tabular} \\[0.7\baselineskip]

    \begin{tabular}{l}
        \prftree[rule]{\scriptsize (!-give)} 
            { \Gamma \parallel \Delta \vdash t : \sigma \Rightarrow \Gamma' \rtimes \beta' }
            { \quad \Gamma' \parallel \Delta \cdot (x : \sigma) \vdash t' : \tau \Rightarrow \Gamma'' \rtimes \beta'' }
            { \quad \operatorname{exp}(\sigma) }
            { \Gamma \parallel \Delta \vdash \textbf{give } x = t \textbf{ in } t' : \tau \Rightarrow \Gamma'' \rtimes \beta' \vee \beta'' }
    \end{tabular} \\[0.7\baselineskip]

    % match
    \begin{tabular}{l}
        % lollipop
        \prftree[rule]{\scriptsize (match-r)} 
            { \Gamma \parallel \Delta \vdash t : \sigma \Rightarrow \Gamma' \rtimes \beta' }
            { \quad \overrightarrow{\Gamma' \parallel \Delta \vdash p_i : \sigma \Uparrow \gamma_i} }
            { \quad \overrightarrow{(\Gamma' \parallel \Delta) \cup \gamma_i \vdash m_i : \tau \Rightarrow \Gamma_i \rtimes \beta_i } }
            { \quad \overrightarrow{\neg \beta_i \implies \gamma_i \cap \Gamma_i = \varnothing } }
            { \Gamma \parallel \Delta \vdash \textbf{match } t \textbf{ with } \overrightarrow{p_i \to m_i} : \tau \Rightarrow \left(\bigcap (\Gamma_i \setminus \gamma_i)\right) \rtimes \left(\beta' \vee \left(\bigvee \beta_i\right)\right) }
    \end{tabular} \\[0.7\baselineskip]

    % \begin{tabular}{l}
    %     % lollipop
    %     \prftree[rule]{\scriptsize (match-r)} 
    %         { \Gamma; \Delta \vdash t : \sigma \Rightarrow \Gamma' }
    %         { \quad \overrightarrow{\Gamma', \Delta \vdash p_i : \sigma \Uparrow \Gamma_i} }
    %         { \quad \overrightarrow{\Gamma_i, \Delta \vdash m_i : \tau \Rightarrow \Gamma_i' } }
    %         { \quad \forall i, j \enskip \Gamma'_i = \Gamma'_j }
    %         { \Gamma; \Delta \vdash \textbf{match } t \textbf{ return } \tau \textbf{ with } \overrightarrow{p_i \to m_i} : \tau \Rightarrow \Gamma'_0 }
    % \end{tabular} \\[1.8\baselineskip]

    \caption{LinML terms typing rules}
    \label{termtyprules}
\end{figure}

\subsubsection{Patterns}


Let $p, p'$ be patterns, $\tau, \sigma$ types, $\Gamma, \Gamma'$ linear typing contexts, and $\gamma, \gamma'$ a multiset of bindings. The typing judgements for patterns have the following shape :
$$
\Gamma \vdash p : \tau \Uparrow \gamma
$$

$\gamma$ contains the bindings that evaluating $p$ added to the environment. \vspace{\baselineskip}

The typing rules are the following:

\begin{figure}[H]
\centering

    % constants & trivialities
    \begin{tabular}{lll}
        % rule for one 
        \prftree[rule]{\scriptsize ($\mathscr P$-$1$)} 
            { \Gamma \vdash * : 1 \Uparrow \varnothing } &
        % rule for variables
        \prftree[rule]{\scriptsize ($\mathscr P$-var)} 
            { \neg \operatorname{exp}(\tau) \implies x \notin \Gamma }
            { \quad \tau \neq \top }
            { \Gamma \vdash x : \tau \Uparrow \{ x : \tau  \} } &
        % rule for tyannot
        \prftree[rule]{\scriptsize ($\mathscr P$-ty)} 
            { \Gamma \vdash p : \tau \Uparrow \gamma }
            { \Gamma \vdash (p : \tau) : \tau \Uparrow \gamma }
    \end{tabular} \\[1.5\baselineskip]

    % with
    \begin{tabular}{ll}
        % rules for with
        \prftree[rule]{\scriptsize ($\mathscr P$-$\&$left)} 
            { \Gamma \vdash p : \sigma \Uparrow \gamma }
            { \Gamma \vdash \langle p, - \rangle : \sigma \with \tau \Uparrow \gamma } &

        \prftree[rule]{\scriptsize ($\mathscr P$-$\&$right)} 
            { \Gamma \vdash p : \tau \Uparrow \gamma }
            { \Gamma \vdash \langle -, p \rangle : \sigma \with \tau \Uparrow \gamma } 
    \end{tabular} \\[1.5\baselineskip]

    % plus
    \begin{tabular}{ll}
        % rules for with
        \prftree[rule]{\scriptsize ($\mathscr P$-$\oplus$left)} 
            { \Gamma \vdash p : \sigma \Uparrow \gamma }
            { \Gamma \vdash (p <: \_ + \tau) : \sigma \oplus \tau \Uparrow \gamma } &

        \prftree[rule]{\scriptsize ($\mathscr P$-$\oplus$right)} 
            { \Gamma \vdash p : \tau \Uparrow \gamma }
            { \Gamma \vdash (p <: \sigma + \_) : \sigma \oplus \tau \Uparrow \gamma }
    \end{tabular} \\[1.5\baselineskip]

    % tensor
    \begin{tabular}{l}
        % rules for with
        \prftree[rule]{\scriptsize ($\mathscr P$-$\otimes$)} 
            { \Gamma \vdash p : \sigma \Uparrow \gamma }
            { \quad \Gamma \vdash p' : \tau \Uparrow \gamma' }
            { \quad \gamma \cap \gamma' = \varnothing }
            { \Gamma \vdash (p, p') : \sigma * \tau \Uparrow \gamma \cup \gamma' }
    \end{tabular} \\[1.5\baselineskip]

    % bang
    \begin{tabular}{ll}
        % rules for bang
        \prftree[rule]{\scriptsize ($\mathscr P$-$\oc$)} 
            { \Gamma \vdash p : \tau \Uparrow \gamma }
            { \Gamma \vdash \oc p : \oc \tau \Uparrow \gamma } & 
        % rules for bang discard
        \prftree[rule]{\scriptsize ($\mathscr P$-$\oc$weaken)} 
            { \Gamma \vdash \_ : \oc \tau \Uparrow \varnothing }
    \end{tabular} \\[1.5\baselineskip]

    % disjunction
    \begin{tabular}{l}
        % rules for bang
        \prftree[rule]{\scriptsize ($\mathscr P$-disj)} 
            { \Gamma \vdash p : \tau \Uparrow \gamma }
            { \quad \Gamma \vdash p' : \tau \Uparrow \gamma' }
            { \quad \gamma = \gamma' }
            { \Gamma \vdash p \mid p' : \tau \Uparrow \gamma } 
    \end{tabular} \\[1.5\baselineskip]



    \caption{LinML pattern typing rules}
    \label{termtyprules}
\end{figure}


\section{LL translation}

\subsection{Prerequisites}

\subsection{Defining the translation}

In all the following, we will only consider the conservative fragment of LinML.

The typing jugement $\Gamma; \Delta \vdash t : \tau \Rightarrow \Gamma'$ has a direct translation in linear logic. The following theorem holds :

$$
\forall \text{ typing jugement } \Gamma; \Delta \vdash t : \tau \Rightarrow \Gamma' \quad \exists \text{ a linear sequent of interface } \Gamma \vdash \otimes (\Gamma' \cup \tau)
$$

\end{document}
