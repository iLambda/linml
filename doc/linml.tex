% various imports 
\documentclass{article}
\usepackage{graphicx}
\usepackage{amsmath}
\usepackage{amsfonts}
\usepackage{amssymb}
\usepackage{mathrsfs}
\usepackage{prftree}
\usepackage{float}
\usepackage{syntax}
\usepackage{cmll}

\usepackage[margin=1in]{geometry}

% define multiset notation
\newcommand*{\ldblbrace}{\{\mskip-5mu\{}
\newcommand*{\rdblbrace}{\}\mskip-5mu\}}

\setlength\parindent{0pt}

\begin{document}

\title{LinML}
\author{Ada Vienot}
\date{}

\maketitle

\section{Typing}

The types in LinML have the following syntax :

\begin{figure}[!h]
    \centering
    \begin{minipage}{0.7\linewidth}
        \setlength{\grammarindent}{3.8em}
        \begin{grammar}
            \let\syntleft\relax
            \let\syntright\relax
            <$\tau, \sigma$> ::= 
                % constant
                $0 \mid 1 \mid \top$ \hfill (type constants)
                % type variables
                \alt $a$ \hfill (type variables)
                % arrows 
                \alt $\tau \multimap \sigma$ \hfill (linear abstraction)
                \alt $\tau \to \sigma$ \hfill (unchecked abstraction)
                % binary constructs 
                \alt $\tau + \sigma$ \hfill (additive disjunction)
                \alt $\tau * \sigma$ \hfill (multiplicative conjunction)
                \alt $\tau \with \sigma$ \hfill (multiplicative disjunction)
                % bang 
                \alt $\tau \oc$ \hfill (of course)
        \end{grammar}
    \end{minipage}
    \caption{LinML types}
    \label{types}
\end{figure}

\subsection{Typing rules}

We define the inductive predicate $\operatorname{exp}$ over types as follows :

$$
\operatorname{exp}(\tau) = 
\left \{
\begin{aligned}
    \mathrm{true} \qquad &\text{if } \tau = * \text{ or } \tau = !\sigma\\
    \mathrm{false} \qquad &\text{elsewise}
\end{aligned}
\right.
$$

In all the following, typing contexts $\Gamma ; \Delta$ are to be seen as a pair with :
\begin{itemize}
    \item $\Gamma$ a multiset of pairs $(x : \tau)$, with $x$ a variable and $\tau$ a type such that $\operatorname{exp}(\tau) = \mathrm{false}$
    
    \item $\Delta$ a set of pairs $(x : \tau)$, with $x$ a variable and $\tau$ a type such that $\operatorname{exp}(\tau) = \mathrm{true}$    
\end{itemize}

Intuitively, $\Gamma$ contains the linear bindings, and $\Delta$ contains the exponential bindings.\vspace{\baselineskip}

Since $\forall \tau, \operatorname{exp}(\tau) = true$ implies the existence of context weakening and contraction rules for $\tau$, we treat the context $\Delta$ as the kind of contexts we manipulate in intuitionistic sequent calculus, that is : they can be duplicated, erased, and one binding is allowed to erase another. This is not the case of the context $\Gamma$ containing linear bindings.

\subsubsection{Terms}

Let $t, u$ be terms, $\tau, \sigma$ types, and $(\Gamma; \Delta), (\Gamma'; \Delta')$ typing contexts. The typing judgements for terms have the following shape :
$$
\Gamma; \Delta \vdash t : \tau \Rightarrow \Gamma'; \Delta'
$$

$\Gamma'; \Delta'$ is the typing context after consuming the necessary bindings to type $t : \tau$. \vspace{\baselineskip}

The typing rules are the following:

\begin{figure}[H]
\centering

    % constants
    \begin{tabular}{ll}
        % rule for one 
        \prftree[rule]{\scriptsize ($1$)} { \Gamma; \Delta \vdash * : 1 \Rightarrow \Gamma; \Delta } &
        % rule for top
        \prftree[rule]{\scriptsize ($\top$)} { \Gamma; \Delta \vdash \langle \rangle : \top \Rightarrow \Gamma; \Delta }
    \end{tabular} \\[\baselineskip]
    % variable 
    \begin{tabular}{ll}
        % variable
        \prftree[rule]{\scriptsize (vlin)} 
            {  (x : \tau) \in \Gamma }
            { \quad \operatorname{exp}(\tau) }
            { \Gamma; \Delta \vdash x : \tau \Rightarrow (\Gamma \setminus (x : \tau)); \Delta }&
        % variable-exp
        \prftree[rule]{\scriptsize (vexp)} 
            {  (x : \tau) \in \Delta }
            { \quad \neg\operatorname{exp}(\tau) }
            { \Gamma; \Delta \vdash x : \tau \Rightarrow \Gamma; \Delta }
    \end{tabular} \\[1.5\baselineskip]

    % times
    \begin{tabular}{l}
        \prftree[rule]{\scriptsize ($\otimes$)} 
            { \Gamma; \Delta \vdash t : \tau \Rightarrow \Gamma'; \_ }
            { \quad \Gamma'; \Delta' \vdash u : \sigma \Rightarrow \Gamma''; \_ }
            { \Gamma; \Delta \vdash (t, u) : \tau * \sigma \Rightarrow \Gamma''; \Delta}
    \end{tabular} \\[1.5\baselineskip]

    % with
    \begin{tabular}{l}
        \prftree[rule]{\scriptsize ($\with$)} 
            { \Gamma; \Delta \vdash t : \tau \Rightarrow \Gamma'; \_ }
            { \quad \Gamma; \Delta \vdash u : \sigma \Rightarrow \Gamma''; \_ }
            { \Gamma' = \Gamma'' }
            % { \Delta' = \Delta'' }
            { \Gamma; \Delta \vdash \langle t, u \rangle : \tau \with \sigma \Rightarrow \Gamma''; \Delta }
            % { \Gamma \vdash t : \tau \Rightarrow \Gamma' }
            % { \quad \Gamma \vdash u : \sigma \Rightarrow \Gamma'' }
            % { (\Gamma \setminus \Gamma') \cap (\Gamma \setminus \Gamma'') = \varnothing }
            % { \Gamma \vdash \langle t, u \rangle : \tau \with \sigma \Rightarrow \Gamma' \cap \Gamma'' }
    \end{tabular} \\[1.5\baselineskip]

    % disjunction
    \begin{tabular}{ll}
        % l 
        \prftree[rule]{\scriptsize ($\oplus$-l)} 
            { \Gamma; \Delta \vdash t : \tau \Rightarrow \Gamma'; \_ }
            { \Gamma; \Delta \vdash (t :> \_ + \sigma) : \tau + \sigma \Rightarrow \Gamma'; \Delta }&
        % r
        \prftree[rule]{\scriptsize ($\oplus$-r)} 
            { \Gamma; \Delta \vdash u : \sigma \Rightarrow \Gamma'; \_ }
            { \Gamma; \Delta \vdash (u :> \sigma + \_) : \tau + \sigma \Rightarrow \Gamma'; \Delta }
    \end{tabular} \\[1.5\baselineskip]

    % bang 
    \begin{tabular}{ll}
        % bang
        \prftree[rule]{\scriptsize ($\oc$)} 
            { \Gamma; \Delta \vdash t : \tau \Rightarrow \Gamma'; \Delta }
            { \Gamma = \Gamma' }
            { \Gamma; \Delta \vdash \oc t : \oc \tau \Rightarrow \Gamma; \Delta  }&
        % % bang
        \prftree[rule]{\scriptsize ($\oc$-dig)} 
            { \Gamma; \Delta \vdash \oc t : \oc \tau \Rightarrow \Gamma'; \Delta }
            { \Gamma; \Delta \vdash \oc \oc t : \oc \tau \Rightarrow \Gamma'; \Delta }
    \end{tabular}\\[1.5\baselineskip]
    
    % \\[0.7\baselineskip]
    % \begin{tabular}{ll}
    %     % bang weak
    %     \prftree[rule]{\scriptsize ($\oc$-weak)} 
    %         { \Gamma \vdash t : \tau }
    %         { \Gamma, x: \oc\sigma \vdash t : \tau } &
    %     % bang contract
    %     \prftree[rule]{\scriptsize ($\oc$-cont)} 
    %         { \Gamma, x: \oc\sigma, x: \oc\sigma \vdash t : \tau }
    %         { \Gamma, x: \oc\sigma \vdash t : \tau }
    % \end{tabular} 
    

    % lolli
    \begin{tabular}{ll}
        % lollipop
        \prftree[rule]{\scriptsize ($\multimap$)} 
            { \Gamma, x:\tau \vdash t : \sigma \Rightarrow \Gamma }
            { \Gamma \vdash \textbf{fun } (x:\tau) \multimap t : \tau \multimap \sigma \Rightarrow \Gamma }&
        % lollipop app
        \prftree[rule]{\scriptsize ($\multimap$-app)} 
            { \Gamma \vdash t : \sigma \multimap \tau}
            { \quad \Gamma' \vdash t' : \sigma }
            { \Gamma, \Gamma' \vdash t t' : \tau }
    \end{tabular} \\[1.8\baselineskip]

    % arrow
    % \begin{tabular}{ll}
    %     % arrow
    %     \prftree[rule]{\scriptsize ($\to$)} 
    %         { \Gamma, x:\oc\tau \vdash t : \sigma }
    %         { \Gamma \vdash \textbf{fun } (x:\tau) \to t : \tau \to \sigma }&
    %     % arrow app 
    %     \prftree[rule]{\scriptsize ($\to$-app)} 
    %     { \Gamma \vdash t : \sigma \to \tau}
    %     { \quad \Gamma' \vdash t' : \oc\sigma }
    %     { \Gamma, \Gamma' \vdash t t' : \tau }
    % \end{tabular}\\[1.5\baselineskip]

    % give
    \begin{tabular}{l}
        \prftree[rule]{\scriptsize (give)} 
            { \Gamma, \Delta, \Delta' \vdash t : \sigma }
            { \quad \Gamma, \Delta, x : \sigma \vdash t' : \tau }
            { \quad (x:\sigma) \notin \Gamma }
            { \Gamma \vdash \textbf{give } x = t \textbf{ in } t' : \tau }
    \end{tabular} \\[0.7\baselineskip]

    \begin{tabular}{l}
        \prftree[rule]{\scriptsize (let)} 
            { \Gamma \vdash \textbf{give } x = !t \textbf{ in } t' : \tau }
            { \Gamma \vdash \textbf{let } x = t \textbf{ in } t' : \tau }
    \end{tabular} \\[1.5\baselineskip]

    \caption{LinML terms typing rules}
    \label{termtyprules}
\end{figure}



\subsubsection{Patterns}

\end{document}
